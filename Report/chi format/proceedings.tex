\documentclass{sigchi}

% Use this command to override the default ACM copyright statement
% (e.g. for preprints).  Consult the conference website for the
% camera-ready copyright statement.

%% EXAMPLE BEGIN -- HOW TO OVERRIDE THE DEFAULT COPYRIGHT STRIP -- (July 22, 2013 - Paul Baumann)
 \toappear{}
%% EXAMPLE END -- HOW TO OVERRIDE THE DEFAULT COPYRIGHT STRIP -- (July 22, 2013 - Paul Baumann)

% Arabic page numbers for submission.  Remove this line to eliminate
% page numbers for the camera ready copy
% \pagenumbering{arabic}

% Load basic packages
\usepackage{balance}
\usepackage{graphicx} % for EPS, load graphicx instead 
\usepackage[T1]{fontenc}
\usepackage{txfonts}
\usepackage{mathptmx}
\usepackage[pdftex]{hyperref}
\usepackage{color}
\usepackage{booktabs}
\usepackage[utf8]{inputenc}
\usepackage{textcomp}
% Some optional stuff you might like/need.
\usepackage{microtype} % Improved Tracking and Kerning
% \usepackage[all]{hypcap}  % Fixes bug in hyperref caption linking
\usepackage{ccicons}  % Cite your images correctly!


% If you want to use todo notes, marginpars etc. during creation of your draft document, you
% have to enable the "chi_draft" option for the document class. To do this, change the very first
% line to: "\documentclass[chi_draft]{sigchi}". You can then place todo notes by using the "\todo{...}"
% command. Make sure to disable the draft option again before submitting your final document.
\usepackage{todonotes}

% Paper metadata (use plain text, for PDF inclusion and later
% re-using, if desired).  Use \emtpyauthor when submitting for review
% so you remain anonymous.
\def\plaintitle{LegoLens: LEGO for the Microsoft Hololens}
\def\plainauthor{First Author, Second Author, Third Author,
  Fourth Author, Fifth Author, Sixth Author}
\def\emptyauthor{}
\def\plainkeywords{Microsoft Hololens; Holotoolkit; LEGO; Augmented reality.}
\def\plaingeneralterms{Documentation, Standardization}

% llt: Define a global style for URLs, rather that the default one
\makeatletter
\def\url@leostyle{%
  \@ifundefined{selectfont}{
    \def\UrlFont{\sf}
  }{
    \def\UrlFont{\small\bf\ttfamily}
  }}
\makeatother
\urlstyle{leo}

% To make various LaTeX processors do the right thing with page size.
\def\pprw{8.5in}
\def\pprh{11in}
\special{papersize=\pprw,\pprh}
\setlength{\paperwidth}{\pprw}
\setlength{\paperheight}{\pprh}
\setlength{\pdfpagewidth}{\pprw}
\setlength{\pdfpageheight}{\pprh}

% Make sure hyperref comes last of your loaded packages, to give it a
% fighting chance of not being over-written, since its job is to
% redefine many LaTeX commands.
\definecolor{linkColor}{RGB}{6,125,233}
\hypersetup{%
  pdftitle={\plaintitle},
% Use \plainauthor for final version.
%  pdfauthor={\plainauthor},
  pdfauthor={\emptyauthor},
  pdfkeywords={\plainkeywords},
  bookmarksnumbered,
  pdfstartview={FitH},
  colorlinks,
  citecolor=black,
  filecolor=black,
  linkcolor=black,
  urlcolor=linkColor,
  breaklinks=true,
}

% create a shortcut to typeset table headings
% \newcommand\tabhead[1]{\small\textbf{#1}}

% End of preamble. Here it comes the document.
\begin{document}

\title{\plaintitle}

\numberofauthors{4}
\author{
  \alignauthor{Thomas Nyegaard-Signori\\
    \email{sfq340@alumni.ku.dk}}\\
  \alignauthor{Tobias Carlos Tvarno\\
    \email{xgp298@alumni.ku.dk}}\\
  \alignauthor{Tor-Salve Dalsgaard\\
    \email{mhb558@alumni.ku.dk}}\\
  \alignauthor{Enes Golic\\
    \email{qzj710@alumni.ku.dk}}\\
}

\maketitle

\begin{abstract}
This project set out to produce an augmented reality application using LEGO on the Microsoft Hololens. Using Unity and the open source toolkit produced by Microsoft called Holotoolkit a prototype was built. Furthermore, the application was tested with AR as a whole kept in mind, and certain negative impacts from this brand new technology were identified and possible solutions were proposed.\\
Two hypothesis were raised concerning the content of the application and the subsequent testing of the application. The proposed solutions need more testing to have a solid foundation, but the first baby steps were taken in what we propose is the right direction. 
\end{abstract}

\keywords{\plainkeywords}

% !TeX root = ../project.tex

\section{Introduction}
\begin{itemize}
\item Teknologi
\item Hvorfor er teknologien mobile?
\item Hvorfor er vores tilgangsvinkel relevant/app ide fed?
\item Hvad var vores allerførste tanker? (måske andet afsnit?)
\end{itemize}

This report focuses on an application for the newly released Microsoft Hololens. The Hololens was released around march 2017, and the fact that the product is in its infant stage and is based on a very new technology opens up possibilities and limits any preconceived notions about which applications and uses the Hololens might have.\\
The technology is relevant with regards to mobile computing in one very apparent way, in that it is a wearable, computing unit. Other than that, it offers alternate reality (AR) possibilities because of its partly see-through screens. Since the Hololens is still a new technology, the mobility and computing power of the product will most likely increase, making it resemble a ubiquitous computer more and more. One of the possible end goals for the Hololens might be integration into wearable lenses or implanted directly into the eyes.\\\\
The application this report will cover revolves around LEGO. LEGO is a way for kids and adults to build constructions, vehicles and scenery, all in a very physical and three-dimensional way. This, then, seemed like a natural choice for an AR application, since the application layer between the user and the world could expand naturally on the possibilities and limitations of the physical, "real-world" LEGO.\\
The application in itself should be a sort of digital playground in which a user could interact with LEGO in ways they would find natural. Sticking pieces together the way they do in real life, stacking and constructing, all interactions that the user knows well from having played around with real LEGO. 

% !TeX root = ../project.tex

\section{Related work}
To ease the interaction between the user and the virtual layer presented by AR, one could think of different interaction techniques, so as to not rely on the gesture recognition of the HoloLens.\\
The implementation with the Hololens presented in \cite{watchsense}. This small, non obtrusive wristband that the prototype emulates could provide another form of interaction, which some users might prefer to the visual tracking of the HoloLens.\\
Another, non-obtrusive way of sending inputs to the HoloLens could be with the use of very small devices using the interaction technique presented in \cite{back}. These devices could act as a secondary way of clicking instead of the "tap" gesture made available by the HoloLens.\\
\\
Providing visual feedback to the user is essential for an AR application to  provide a meaningful experience, but since the line is blurred between real world and virtual world, the question of natural feedback remains. This is presented in both \cite{strøm} and \cite{finger}, though in wildy different approaches. \cite{strøm} approaches the natural feedback by means of kinesthetic feedback and \cite{finger} approaches it by means of haptic feedback. Using these modalities could provide the HoloLens with a more natural feeling when interacting with the objects in the virtual world.\\
\\
A possible extra connection between the virtual and real world could be by including and extending real world objects into the virtual world. This is done in \cite{multifi}, and could help with creating a link between the two worlds. 

% !TeX root = ../project.tex

\section{Design decisions}
During the sketching phase, several design choices were discussed. One of the very first decisions made was the overall look of the menu. The choice of big buttons, clear visual cues and short textual descriptions was present in almost all of the sketches in the early design phase, as seen in figure (HVILKEN FIGURE???):\\
\begin{figure}
\centering
  \includegraphics[width=0.9\columnwidth]{figures/Menu/menu2.png}
  \caption{Skriv lidt lækert her. }~\label{fig:genboard}
\end{figure}
\begin{figure}
\centering
  \includegraphics[width=0.6\columnwidth]{figures/Menu/menu1.png}
  \caption{Skriv lidt lækert her. }~\label{fig:genboard}
\end{figure}

\subsection{Suitability of LEGO in AR}
One of the aspects with the application was the suitability of LEGO in an AR setting. This question was asked rather late in the development process. What became apparent was that the virtual LEGO in an AR setting would not be able to provide the same "finicky" feel that LEGO has, sitting at a table, obsessing over small details in an advanced (?) setup. This is because of the computing limitiations of the Hololens and the technology architecture. The minimal rendering distance of the Hololens is, right now, much larger than the distance one would be from af real-life LEGO project, ie., arms length. This limitation demands a much larger brick size than real-life LEGO and this in turn limits the overall complexity of a virutal LEGO project. 

% !TeX root = ../proceedings.tex
\section{Implementation}

\subsection{Language and tools}
The LegoLens application is built using Unity, where custom scripts are written in C\#. Unity is chosen, since it is used in Microsofts own Hololens academy.\footnote{\url{https://developer.microsoft.com/en-us/windows/mixed-reality/academy}} Microsoft also provides a software toolkit called HoloToolkit\footnote{\url{https://github.com/Microsoft/HoloToolkit-Unity}}, which binds Unity and the Hololens together. \\
For initial testing the Microsoft Hololens Emulator was used, while in later stages the application was deployed directly to the Hololens device. 

\subsection{HoloToolkit}
The HoloToolkit provides implementations of common tasks in developing applications for the Hololens. This kit implements functionality, which is mostly Hololens specific, such as spatial mapping and understanding. Also functionality for input methods are provided. The Hololens comes with two build-in gestures: the bloom and the tap.

\subsubsection{Spatial mapping and understanding}
These two functionalities are fundamental for the use of the Hololens, since they let the Hololens understand the shape of a room. Thereby virtual objects can be placed in the real world. The toolkit provides prefabricated functionality for spatial mapping and understanding, which simply can be added in Unity.  \\
The Hololens uses a depth camera similar to the camera build into the Kinect v2 and four cameras meant to understand environment.\footnote{\url{https://developer.microsoft.com/en-us/windows/mixed-reality/hololens_hardware_details}} The spatial mapping relies on the time-of-flight Kinect-like depth camera and the four RGB cameras to provide a robust tracking of the enviroment. \\
This connection between the real-world and digital playground created in the application was essential, both to the experience but also to make the application augmenting the reality. The mapping allows to create a relative coordinate system, such that virtual objects can be placed in accurately.

\subsubsection{Input methods}
The Hololens tracks the gaze of the wearer, by the assumption that the wearer always looks straight ahead. This is the limitations of the Hololens, which does not provide augmentation for the whole field of view. Thus if the wearers eyes would be tracked, the wearer could gaze outside of the Hololens field of view. In this application a cursor appears where the gaze of the wearer collides with mapped objects in the real world or objects created by the application.\\ 
In this application the tap method is used. Here there are two different states the interaction can have: a single tap and dragging. \\
The single tap is used to navigate in the menu on the generator. The HoloToolkit provides functionality, such that objects can be dragged by holding tap. This is used to drag LEGO bricks and the generator around.

\subsection{Unity scripts}
While the HoloToolkit implements the augmentation of objects, the custom Unity scripts implement LEGO functionality. These scripts control the behaviour of the LEGO bricks, the generator and templates. 

\subsubsection{Generator}
The generator script contains the main functionality of the implemented code. Menu interactions are implemented here and thus the necessary functions to instantiate and model bricks. The instantiation is done by loading the prefabricated LEGO brick object, and colouring it depending on user input. A brick is not affected by gravity until the user interacts with the brick. For the purpose of initial simplicity the number of shapes a brick can have is limited. \\
Templates are implemented in a similar way as bricks, except that these can not be coloured. Also the sandbox feature is implemented in the generator script. It generates 20 bricks with random shape and colour. \\
The menu itself on the generator works by activating and deactivating the appropriate menu items.

\subsubsection{Bricks and Templates}
As described above, the brick and template scripts work in a similar way. They both make sure that the generated objects do not rotate and do snap to a grid. A one-by-one brick is 0.8cm long and wide and 0.96cm tall (with knobs). Thus bricks are forced to be placed in positions which are multiples of 0.8.

% !TeX root = ../proceedings.tex

\section{Evaluation}
The goal of the test was to test two hypotheses regarding HoloLens and the application:
\begin{itemize}
	\item[\textit{\textbf{H1}:}] The user will be confused/overwhelmed by the virtual layer that the HoloLens provides
	\item[\textit{\textbf{H2}:}] LEGO is well suited for the HoloLens
\end{itemize}
We had two test participants, male and female from different institutions, both of age 23. Chocolate was offered as a reward but was not taken. These were then put through a three-part test, the first part acted as a training session for the actual interaction with the HoloLens application, the second one was different tests in the application wearing the HoloLens and the last part was a short questionnaire.\\
\\
Since we wanted to test whether or not the HoloLens was too much of an overwhelming experience for the user, the tests were structured so as to minimize any complications with the application itself. Furthermore, the subjects had never been in contact with any AR or virtual reality interfaces beforehand.\\
The first part, the training session, was created using Prototype on Paper\footnote{\url{https://marvelapp.com/pop/}} and physical representations of LEGO bricks. The setup was meant to emulate the view in the HoloLens app as much as possible whilst keeping the overall test setup as low-fidelity as possible. The users were asked to go through most of the menus functionality in short, manageable tasks which were meant to prime them for the same tasks in the HoloLens application.\\
The second part of the test was in the application itself after a brief introduction to gestures using the HoloLens app "Learn Gestures". This was done to give the test subjects a complete overview of the interaction possibilities with the HoloLens. The test subjects were asked to do simple tasks, such as spawning a brick of a chosen color and shape or instantiating a template.\\
The third and final part was a short questionnaire consisting of Likert scales and free text questions. These questions were made with the hypotheses in mind without imposing a bias on the subjects.\\
\\
The evaluation of the results were done in a non-formal way as opposed to the statistical methods described in the lectures (ie. ANOVA, Kruskal-Walls Test etc.). This was done due to our tests not being of a comparative nature, and thus we did not find any way of imposing a strict statistical test on our (small!) test sample. Our test results are based solely on the subjects experiences and opinions and are therefore to be taken with a grain of salt. This test structure came to be because of our hypotheses.

\subsection{Results}
All the Likert scale questions were done on a 1-5 scale.\\
Both participants found the menu very easy to navigate (both gave a 5 on the Lickert scale), and they both agreed on the application/HoloLens being able to track their movements and position satisfyingly. The subjects disagreed on the precision of the placement of the brick, and when asked about this precision one subject said that he/she did not feel comfortable with the interaction since he/she had experienced beforehand that bricks and generators could fall through solid, real world objects.\\

\subsection{Observations}
One user noted that the HoloLens' tracking was thrown off balance when looking into lights, and this confused him/her. Both users felt dizzy when they took of the HoloLens and complained about a queeziness when having the HoloLens on for 15 minutes or more.\\
Both users thought that a LEGO application would suit the HoloLens/AR in general, because of the possibilities, although the size of the bricks and projects as a whole would have to be considerably larger than real world LEGO bricks. Both participants agreed the size of the bricks presented in the high fidelity prototype was appropriate, although they were not presented with alternatives.\\
It was observed that one of the subjects, immediately after dropping the generator on the ground, used the "bloom" gesture instead of looking down, since he/she thought that the application had crashed. This could indicate that H1 has relevance and should be investigated further.\\
One user noted that he/she was unsure whether the control of the brick was done using the gaze or finger dragging.

% !TeX root = ../project.tex

\section{Discussion}
\subsection{Suitability of LEGO in AR}
One of the aspects with the application was the suitability of LEGO in an AR setting. This question was asked rather late in the development process. What became apparent was that the virtual LEGO in an AR setting would not be able to provide the same "finicky" feel that LEGO has, sitting at a table, obsessing over small details in an advanced (?) setup. This is because of the computing limitiations of the Hololens and the technology architecture. The minimal rendering distance of the Hololens is, right now, much larger than the distance one would be from af real-life LEGO project, ie., arms length. This limitation demands a much larger brick size than real-life LEGO and this in turn limits the overall complexity of a virutal LEGO project. 

% !TeX root = ../project.tex

\section{Future work}
\subsection{LEGO size and compexity}
To combat the issue with the size of actual LEGO and the complexity available in this scale a virtual magnifying glass could be envisioned. Using a special gesture, a certain area of the LEGO bricks could be brought into view using a secondary "screen", a menu that could show a zoomed in version of the actual view the user has. (Er det lort?)

%Kan bruges måske (http://www.mitpressjournals.org/doi/abs/10.1162/pres.1997.6.4.399) ???

\subsection{Brick shortage}
We have a severe shortage of bricks available in our prototype. Anyone who considers an AR LEGO application will soon have to think of which blockset or types they would include. The amount of distinctive LEGO bricks is staggeringly high, and this sort of work would probably benefit greatly from working together with LEGO as to get dimensions and oddities right. This work would also make such an application much more attractive, as one of the strongest selling points of LEGO is the variety but ensured compatibility. 

%Antallet af forskellige LEGO bricks der findes, som lol stat https://www.quora.com/How-many-types-of-LEGO-bricks-parts-are-there

% !TeX root = ../project.tex

\section{Conclusion}
The application presented in this report was equal parts a technical implementation as well as an experiment in how to test something as potentially overwhelming as augmented reality. We have discussed possible problems with testing subjects which have both the specific application to be tested as well as AR as completely new technological inputs and possible solutions to these problems.\\
In future work, more (thorough) testing needs to be done and a best practice as to how to test complete beginners in an application utilizing augmented reality. Furthermore, the LEGO application itself needs more bricks and functionality to ease the interaction with said bricks. \\
\\
Concerning the two hypotheses presented in the paper, it was found that users had a difficult time differentiating between application difficulties and limitations of the HoloLens, indicating that H1, the hypothesis revolving the virtual layer and the "wow-efffect", might hold with regards to AR applications.\\
H2 showed promise, as subjects were positive towards the included functionality, however rough it may have been implemented in the high fidelity prototype. 

\nocite{*}

% REFERENCES FORMAT
% References must be the same font size as other body text.
\bibliographystyle{SIGCHI-Reference-Format}
\bibliography{sample}

\end{document}

%%% Local Variables:
%%% mode: latex
%%% TeX-master: t
%%% End:
