% !TeX root = ../project.tex

\section{Conclusion}
The application presented in this report was equal parts a technical implementation as well as an experiment in how to test something as potentially overwhelming as augmented reality. We have discussed possible problems with testing subjects which have both the specific application to be tested as well as AR as completely new technological inputs and possible solutions to these problems.\\
In future work, more (thorough) testing needs to be done and a best practice as to how to test complete beginners in an application utilizing augmented reality. Furthermore, the LEGO application itself needs more bricks and functionality to ease the interaction with said bricks. \\
\\
Concerning the two hypotheses presented in the paper, it was found that users had a difficult time differentiating between application difficulties and limitations of the HoloLens, indicating that H1, the hypothesis revolving the virtual layer and the "wow-efffect", might hold with regards to AR applications.\\
H2 showed promise, as subjects were positive towards the included functionality, however rough it may have been implemented in the high fidelity prototype. 