% !TeX root = ../proceedings.tex

\section{Related work}
To ease the interaction between the user and the virtual layer presented by AR, one could think of different interaction techniques, so as to not rely on the gesture recognition of the HoloLens.\par
The implementation with the Hololens presented in \cite{watchsense}. This small, non obtrusive wristband that the prototype emulates could provide another form of interaction, which some users might prefer to the visual tracking of the HoloLens.\par
Another, non-obtrusive way of sending inputs to the HoloLens could be with the use of very small devices using the interaction technique presented in \cite{back}. These devices could act as a secondary way of clicking instead of the "tap" gesture made available by the HoloLens. Also using the skin itself as a possible interaction surface, as presented in \cite{skin}, could lead to and expanded interaction palette.\par
Providing visual feedback to the user is essential for an AR application to  provide a meaningful experience, but since the line is blurred between real world and virtual world, the question of natural feedback remains. This is presented in both \cite{stroem} and \cite{finger}, though in wildy different approaches. \cite{stroem} approaches the natural feedback by means of kinesthetic feedback and \cite{finger} approaches it by means of haptic feedback. Using these modalities could provide the HoloLens with a more natural feeling when interacting with the objects in the virtual world.\par
A possible extra connection between the virtual and real world could be by including and extending real world objects into the virtual world. This is done in \cite{multifi}, and could help with creating a link between the two worlds. 