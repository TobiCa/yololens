% !TeX root = ../project.tex

\section{Future work}
\subsection{LEGO size and compexity}
To combat the issue with the size of actual LEGO and the complexity available in this scale a virtual magnifying glass could be envisioned. Using a special gesture, a certain area of the LEGO bricks could be brought into view using a secondary "screen", a menu that could show a zoomed in version of the actual view the user has. (Er det lort?)

%Kan bruges måske (http://www.mitpressjournals.org/doi/abs/10.1162/pres.1997.6.4.399) ???

\subsection{Brick shortage}
We have a severe shortage of bricks available in our prototype. Anyone who considers an AR LEGO application will soon have to think of which blockset or types they would include. The amount of distinctive LEGO bricks is staggeringly high, and this sort of work would probably benefit greatly from working together with LEGO as to get dimensions and oddities right. This work would also make such an application much more attractive, as one of the strongest selling points of LEGO is the variety but ensured compatibility. 

%Antallet af forskellige LEGO bricks der findes, som lol stat https://www.quora.com/How-many-types-of-LEGO-bricks-parts-are-there