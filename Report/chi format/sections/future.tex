% !TeX root = ../proceedings.tex

\section{Future work}

\subsection{Hypothesis specific}
To work with these hypothesis, more tests would have to be conducted, but also the application stability would have to improve dramatically. Users have to feel confident in the application so as to reveal any problems with AR interaction methods as a whole, so that any frustrations and considerations are directed against AR instead of AR AND the application.\\
\\
With regards to H2, tests showed that brick size, ease of access through the menu and a general handholding of the user (in our case, bricks being locked in rotation and snapping to a virtual grid) improved the LEGO experience in an AR setting. Since the technology is not fully developed it is difficult to specify best parameters such as brick size and functionalities that would improve precision. The test subjects showed a general consensus in that LEGO was suitable for AR. One problem is that they only had one point of reference, our high fidelity, but unfinished prototype, and thus it becomes a very subjective matter whether the brick size is actually of a appropriate size or that LEGO as a whole makes sense in AR.\\
One way of narrowing down specific parameters would be a comparative approach to testing, ie. a prototype with different bricks sizes, functionalities and input methods. 

\subsection{Application specific}
To combat the issue with the size of actual LEGO and the complexity available in this scale a virtual magnifying glass could be envisioned. Using a special gesture, a certain area of the LEGO bricks could be brought into view using a secondary "screen", a menu that could show a zoomed in version of the actual view the user has, such as a magnifying glass.\\
\\
The number of input methods implemented into the HoloLens is limited and this is what a test subject criticized. Thus an extension of the input methods is an obvious next step to take. Here the choice of extensions ranges from implementing more hand gestures to using methods such as WatchSense \cite{watchsense}, which utilizes SmartWatches and gestures on the back of the hand to provide a new interaction method. \\
\\
There is a severe shortage of bricks available in the prototype. Anyone who considers an AR LEGO application will soon have to think of which set of blocks or types they would include. The amount of distinctive LEGO bricks is staggeringly high, and this sort of work would probably benefit greatly from working together with LEGO as to get dimensions and oddities right. This work would also make such an application much more attractive, as one of the strongest selling points of LEGO is the variety, but ensured compatibility.\\
\\
Improving the stability of the spatial mapping would have to be a top priority if the prototype is to evolve from the state that it is in at the moment. To many disappearing bricks and generators end up with confused users, which is very troublesome taking H1 into consideration. 