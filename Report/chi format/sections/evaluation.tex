% !TeX root = ../proceedings.tex

\section{Evaluation}

participants
test setting
	- hypothesis
evaluation

The goal of the test was to test two hypothesis regarding HoloLens and the application:
\begin{itemize}
	\item[\textit{\textbf{H1}:}] The user will be confused/overwhelmed by the virtual layer that the HoloLens provides
	\item[\textit{\textbf{H2}:}] LEGO is well suited for the HoloLens
\end{itemize}
We had two test participants, male and female from different institutions, both of age 23. Chocolate was offered as a reward but was not taken. These were then put through a three-part test, the first part acted as a training session for the actual interaction with the HoloLens application, the second one was different tests in the application wearing the HoloLens and the last part was a short questionnaire.\\
\\
Since we wanted to test whether or not the HoloLens was too much of an overwhelming experience for the user, the tests were structured so as to minimize any complications with the application itself. Furthermore, the subjects had never been in contact with any AR or virtual reality interfaces beforehand.\\
The first part, the training session, was created using Prototype on Paper\footnote{\url{https://marvelapp.com/pop/}} and physical representations of LEGO bricks. The setup was meant to emulate the view in the HoloLens app as much as possible whilst keeping the overall test setup as low-fidelity as possible. The users were asked to go through most of the menus functionality in short, manageable task which were meant to prime them for the same task in the HoloLens application.\\
The second part of the test was in the application itself after a brief introduction to gestures using the HoloLens app "Learn Gestures". This was done to give the test subjects a complete overview of the interaction possibilities with the HoloLens. The test subjects were asked to do simple tasks, such as spawning a brick of a chosen color and shape or instantiating a template.\\
The third and final part was a short questionnaire consisting of Likert scales and free text questions. These questions were made with the hypothesis in mind without imposing a bias on the subjects.\\
\\
The evaluation on the results were done in a non-formal way as opposed to the statistical methods described in the lectures (ie. ANOVA, Kruskal-Walls Test etc.). This was done due to our tests not being of a comparative nature, and thus we did not find any way of imposing a strict statistical test on our (small!) test sample. Our test results are based solely on the subjects experiences and opinions and are therefore to be taken with a grain of salt. This test structure came to be because of our hypothesis.

\subsection{Results}
All the Likert scale questions were done on a 1-5 scale.\\
Both participants found the menu very easy to navigate (both gave a 5 on the Lickert scale), and they both agreed on the application/HoloLens being able to track their movements and position satisfyingly. The subjects disagreed on the precision of the placement of the brick, and when asked about this precision one subject said that he/she did not feel comfortable with the interaction since he/she had experienced beforehand that bricks and generators could fall through solid, real world objects.\\

\subsection{Observations}
One user noted that the HoloLens' tracking was thrown off balance when looking into lights, and this confused him/her unnecessarily. Both users felt dizzy when they took of the HoloLens and complained about a queeziness when having the HoloLens on for 15 minutes or more.\\
Both user thought that a LEGO application would suit the HoloLens/AR in general, because of the possibilities, although the size of the bricks and projects as a whole would have to be considerably larger than real world LEGO bricks. Both participants agreed on the size of the bricks presented in the high fidelity prototype was appropriate, although they were not presented with alternatives.\\
It was observed that one of the subjects, immediately after dropping the generator on the ground, used the "bloom" gesture instead of looking down, since he/she thought that the application had crashed. This could indicate that H1 has relevance and should be investigated further.\\
One user noted that he/she was unsure whether the control of the brick was done using the gaze or finger dragging.