% !TeX root = ../project.tex

\section{Introduction}
This report focuses on an application for the newly released Microsoft Hololens. The Hololens was released around march 2017, and the fact that the product is in its infant stage and is based on a very new technology opens up possibilities and removes any preconceived notions about which applications and uses the Hololens might have.\\
The technology is relevant with regards to mobile computing in one very apparent way, in that it is a wearable, computing unit. Other than that, it offers alternate reality (AR) possibilities because of its partly see-through screens and user tracking. Since the Hololens is still a new technology, the mobility and computing power of the product will most likely increase, making it resemble a ubiquitous computer more and more. This evolution of mobile computing was one of the reasons that the Hololens was chosen to develop on in the first place.\\\\
The application this report will cover revolves around LEGO. LEGO is a way for kids and adults to build constructions, vehicles and scenery, all in a very physical and three-dimensional way. This, then, seemed like a natural choice for an AR application, since the application layer between the user and the world could expand naturally on the possibilities and limitations of the physical, "real-world" LEGO.\\
The application in itself should be a sort of digital playground in which a user could interact with LEGO in ways they would find natural. Sticking pieces together the way they do in real life, stacking and constructing, all interactions that the user knows well from having played around with real LEGO. This was done using interaction through a virtual tablet, known as the "Generator board" and simple drag-and-drop with the bricks. We end up with a rough prototype which helped us discover the pitfalls and consideration concerning a LEGO implementation in an AR enviroment. 