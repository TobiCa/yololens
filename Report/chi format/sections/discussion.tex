% !TeX root = ../project.tex

\section{Discussion}
\subsection{Suitability of LEGO in AR}
One of the aspects with the application was the suitability of LEGO in an AR setting. This question was asked rather late in the development process. What became apparent was that the virtual LEGO in an AR setting would not be able to provide the same "finicky" feel that LEGO has, sitting at a table, obsessing over small details in an advanced (?) setup. This is because of the computing limitiations of the Hololens and the technology architecture. The minimal rendering distance of the Hololens is, right now, much larger than the distance one would be from a real-life LEGO project, ie., arms length. This limitation demands a much larger brick size than real-life LEGO and this in turn limits the overall complexity of a virutal LEGO project. \\
These considerations became our main focus in the user tests, whether or not the HoloLens is applicable in these small scale, high detail applications such as LEGO. High detail, in this sense, refers to the many ways that LEGO bricks can be joined together to create more and more complex structures. With these structures comes problems, such as occlusions, tracking precision and stabilization issues. 