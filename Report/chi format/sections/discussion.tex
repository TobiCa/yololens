% !TeX root = ../project.tex

\section{Discussion}

\subsection{Confusion and overwhelming impressions (H1)}
One aspect of the conducted tests was the apparent wow-factor of the HoloLens and virtual layer. The tests and observations showed that the subjects could not utilize the fact that the virtual layer extended beyond their apparent field-of-view. This resulted in many "out of sight, out of mind" situations, where the subjects tried to restart the application although nothing critical had happened. This can of course be explained and mediated using markers and other helping layers to indicate where the points of interest in the application may be located, but this would then in turn clutter and already sparse field-of-view. To ease the subjects into thinking in an AR setting we used the already existing "Learn Gestures" application, but this did not seem to have a strong beneficial impact on that issue.\\
The issue that H1 covers is under discussion in \cite{kim}. The results in \cite{kim} show that user studies need to be central in the development of AR as well as virtual reality applications. 

\subsection{Suitability of LEGO in AR (H2)}
One of the hypothesis introduced in the previous section was the suitability of LEGO in an AR setting. This question was asked rather late in the development process. What became apparent was that the virtual LEGO in an AR setting would not be able to provide the same "finicky" feel that LEGO has, sitting at a table, obsessing over small details in an advanced setup. This is because of the computational limitations of the HoloLens and the technology architecture. The minimal rendering distance of the HoloLens is, right now, much larger than the distance one would be from a real-life LEGO project, ie., arms length. This limitation demands a much larger brick size than real-life LEGO and this in turn limits the overall complexity of a virtual LEGO project. \\
These considerations became our main focus in the user tests, whether or not the HoloLens is applicable in these small scale, high detail applications such as LEGO. High detail, in this sense, refers to the many ways that LEGO bricks can be joined together to create more and more complex structures. With these structures comes problems, such as occlusions, tracking precision and stabilization issues. 