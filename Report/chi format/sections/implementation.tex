% !TeX root = ../project.tex
\section{Implementation}

\subsection{Unity}
The application was built using Unity and mainly written in C#. Unity provides a real-time interaction with the scenes in an application and has tools to ease the development of AR applications. (FUCKING SKRIV MERE).

\subsection{Holotoolkit}
The main package used for developing for the Hololens in this project was the HoloToolkit-Unity. This package comes with some premade functionality to ease the interaction with the Hololens and is an open source toolkit made by Microsoft to speed up any development for their new platform (quote dem måske?).\\
\\
The toolkit contains seven feature areas, of those spatial mapping and input were of most interest to us. We used the spatial mapping part of the toolkit to be able to "digitize" the world and make our application able to track surfaces so as to place our generator board on real world surfacs instead of having it float in mid air. This connection between the real-world and digital playground created in our application was essential, both to the experience but also to be able to call our application alternate reality.\\
The spatial mapping relies on the time-of-flight depth cameras and RGB cameras to provide a robust tracking of the enviroment. This mapping is then readily available to developers through the Holotoolkit and can be applied to object in an application.
(SKRIV MERE TEKNISK NÅR VI VED HVAD FUCK DER FOREGÅR) 
\\\\
The input part of the toolkit allows us to track the gaze and the users interaction with the objects in the application, be it buttons, bricks or the generator board. This tracking is done by shooting a ray from the users gaze (the middle of the screen on the Hololens in our case) and checking whether any colliders, object hitboxes, were hit. This raycasting, as its called, is intuitive since it uses the line of sight from the user to any object in the gameworld, so whatever you can see, you can interact with.\\
More specifically, the orientation and position of the Hololens with regards to the objects in the gameworld is maintained by a GazeManager from the HoloToolkit and the cursor is then placed on the vector originating from the users gaze by a CursorManager. This raycast depends on the Hololens ability to track the user using gyroscopes, accelerometers and computer vision. By tracking the user gaze in the real world and imposing a mapping from the real-world to a virtual-world coordinate system, the user can be mapped in with reference to the virtual objects (QUOTE EN TEKST, FÅ DET TIL AT LYDE MINDRE OSTET). 