\documentclass[11pt]{article}
\usepackage[a4paper, hmargin={2.8cm, 2.8cm}, vmargin={2.5cm, 2.5cm}]{geometry}
\usepackage{eso-pic} % \AddToShipoutPicture
\usepackage{graphicx} % \includegraphics
\usepackage{amsmath}
\usepackage{amsfonts}
\usepackage{amssymb}
\usepackage{graphicx}
\usepackage{fancyhdr}
\usepackage{moreverb}
\usepackage{lscape}
\usepackage[utf8]{inputenc}
\usepackage{caption}
\usepackage{subcaption}
\usepackage{float}

% Ved at bruge kommandoen \newcommand kan man forkorte kommandoer eller ændre dem til noget mere passende.
\newcommand{\setR}{\mathbb{R}}
\newcommand{\setZ}{\mathbb{Z}}
\newcommand{\setN}{\mathbb{N}}
\newcommand{\setF}{\mathbb{F}}
\newcommand{\lra}{\Leftrightarrow}
\newcommand{\ra}{\Rightarrow}
\newcommand{\ac}{\textasciicircum}
\newcommand{\uuline}[1]{\underline{\underline{#1}}}
\newcommand{\bpm}{\begin{pmatrix}}
\newcommand{\epm}{\end{pmatrix}}

\renewcommand{\headrulewidth}{0pt}

\author{
  \Large{}
  \\ \texttt{} \\
}

\title{Mobile computing - Hololens and LEGO
  \vspace{3cm}
  \Huge{} \\
  \Large{Thomas Nyegaard-Signori\\
  Enes Golic\\
  Tor-Salve Dalsgaard\\
  Tobias Carlos Tvarnø}
}

\begin{document}

%% Change `ku-farve` to `nat-farve` to use SCIENCE's old colors or
%% `natbio-farve` to use SCIENCE's new colors and logo.
\AddToShipoutPicture*{\put(0,0){\includegraphics*[viewport=0 0 700 600]{include/ku-farve}}}
\AddToShipoutPicture*{\put(0,602){\includegraphics*[viewport=0 600 700 1600]{include/ku-farve}}}

%% Change `ku-en` to `nat-en` to use the `Faculty of Science` header
\AddToShipoutPicture*{\put(0,0){\includegraphics*{include/ku-en}}}

\clearpage\maketitle
\thispagestyle{empty}

\newpage

\section{Introduction}
\begin{itemize}
\item Teknologi
\item Hvorfor er teknologien mobile?
\item Hvorfor er vores tilgangsvinkel relevant/app ide fed?
\item Hvad var vores allerførste tanker? (måske andet afsnit?)
\end{itemize}

This report focuses on an application for the newly released Microsoft Hololens. The Hololens was released in march 2017, and the fact that the product is in its infant stage and is still very much a new technology opens up the possibilities and limits any preconceived notions about which applications and uses the hololens might have.\\
The technology is relevant with regards to mobile computing in one very apparent way, in that it is a wearable, computing unit. Other than that, it offers alternate reality (AR) possibilities because of its partly see-through screens. Since the Hololens is still a new technology, the mobility of the product will most likely increase, making it resemble a ubiquitous computer more and more, as the end goal of the Hololens might be integration into wearable lenses and directly into the eyes. 

\section{Sketching}
The structure can look like this:
\begin{itemize}
	\item Present the chosen design challenges and some of the initial concerns in the beginning phase
	\item Each challenge is outlined with:
		\begin{itemize}
			\item Sketches
			\item Design thoughts for the sketch - what was the thought behind this sketch?
			\item summary of the group discussion for this case - what worked, what didn't and why?
		\end{itemize}
	\item Conclude what the final sketches are - what do they accomplish, and what do they lack(maybe reference what might be discussed in future work)?
\end{itemize}
\subsection{How do we access the application? (Menu)}
The menu is an essential part of any application. It outlines the possibilities for the user in a simple fashion. The menu is something every user has experience with, as it is the first thing a user is met with when running an application. This means that the menu has to contain of certain classic elements.\par A user needs a way to close the application. On mobile phones nowadays this can be done with 'return' buttons on the phone, but all applications generally have a built in exit function.\par
The user also needs to have some sort of options menu, and guidelines. This is a must for this application. Stacking LEGO seems simple and intuitive, but all the operations and possibilities is something that can confuse a potential user. \par
Lastly the menu needs to have an easy access to the LEGO session itself. It shouldn't be complicated for the user to start a new LEGO session, meaning that this action should be as intuitive as it gets.\par
\begin{center}
	Insert sketch 1:
\end{center}

%\begin{figure}[H]
%    \centering
%   \begin{subfigure}[b]{0.4\textwidth}
%        \includegraphics[width=\textwidth]{./billeder/6-kode.PNG}
%        \caption{Code for Wiener filter generation.}
%        \label{fig:4-pic}
%   \end{subfigure}
%        \begin{subfigure}[b]{0.5\textwidth}
%        \includegraphics[width=\textwidth]{./billeder/6-pic.PNG}
%        \caption{Images with the different Wiener parameters.}
%        \label{fig:4-pic}
%    \end{subfigure}
%    \caption{Exercise 6}
%\end{figure}

\end{document}
