\documentclass[11pt]{article}
\usepackage[a4paper, hmargin={2.8cm, 2.8cm}, vmargin={2.5cm, 2.5cm}]{geometry}
\usepackage{eso-pic} % \AddToShipoutPicture
\usepackage{graphicx} % \includegraphics
\usepackage{amsmath}
\usepackage{amsfonts}
\usepackage{amssymb}
\usepackage{graphicx}
\usepackage{fancyhdr}
\usepackage{moreverb}
\usepackage{lscape}
\usepackage[utf8]{inputenc}
\usepackage{caption}
\usepackage{subcaption}
\usepackage{float}

% Ved at bruge kommandoen \newcommand kan man forkorte kommandoer eller ændre dem til noget mere passende.
\newcommand{\setR}{\mathbb{R}}
\newcommand{\setZ}{\mathbb{Z}}
\newcommand{\setN}{\mathbb{N}}
\newcommand{\setF}{\mathbb{F}}
\newcommand{\lra}{\Leftrightarrow}
\newcommand{\ra}{\Rightarrow}
\newcommand{\ac}{\textasciicircum}
\newcommand{\uuline}[1]{\underline{\underline{#1}}}
\newcommand{\bpm}{\begin{pmatrix}}
\newcommand{\epm}{\end{pmatrix}}

\renewcommand{\headrulewidth}{0pt}

\author{
  \Large{}
  \\ \texttt{} \\
}

\title{Mobile computing - Hololens and LEGO
  \vspace{3cm}
  \Huge{} \\
  \Large{Thomas Nyegaard-Signori\\
  Enes Golic\\
  Tor-Salve Dalsgaard\\
  Tobias Carlos Tvarnø}
}

\begin{document}

%% Change `ku-farve` to `nat-farve` to use SCIENCE's old colors or
%% `natbio-farve` to use SCIENCE's new colors and logo.
\AddToShipoutPicture*{\put(0,0){\includegraphics*[viewport=0 0 700 600]{include/ku-farve}}}
\AddToShipoutPicture*{\put(0,602){\includegraphics*[viewport=0 600 700 1600]{include/ku-farve}}}

%% Change `ku-en` to `nat-en` to use the `Faculty of Science` header
\AddToShipoutPicture*{\put(0,0){\includegraphics*{include/ku-en}}}

\clearpage\maketitle
\thispagestyle{empty}

\newpage

\section{Introduction}
\begin{itemize}
\item Teknologi
\item Hvorfor er teknologien mobile?
\item Hvorfor er vores tilgangsvinkel relevant/app ide fed?
\item Hvad var vores allerførste tanker? (måske andet afsnit?)
\end{itemize}

This report focuses on an application for the newly released Microsoft Hololens. The Hololens was released in march 2017, and the fact that the product is in its infant stage and is still very much a new technology opens up the possibilities and limits any preconceived notions about which applications and uses the hololens might have.\\
The technology is relevant with regards to mobile computing in one very apparent way, in that it is a wearable, computing unit. Other than that, it offers alternate reality (AR) possibilities because of its partly see-through screens. Since the Hololens is still a new technology, the mobility of the product will most likely increase, making it resemble a ubiquitous computer more and more, as the end goal of the Hololens might be integration into wearable lenses and directly into the eyes. 

\section{Sketching}

%\begin{figure}[H]
%    \centering
%   \begin{subfigure}[b]{0.4\textwidth}
%        \includegraphics[width=\textwidth]{./billeder/6-kode.PNG}
%        \caption{Code for Wiener filter generation.}
%        \label{fig:4-pic}
%   \end{subfigure}
%        \begin{subfigure}[b]{0.5\textwidth}
%        \includegraphics[width=\textwidth]{./billeder/6-pic.PNG}
%        \caption{Images with the different Wiener parameters.}
%        \label{fig:4-pic}
%    \end{subfigure}
%    \caption{Exercise 6}
%\end{figure}

\end{document}
